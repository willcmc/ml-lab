\documentclass[a4paper,12pt]{report}
\usepackage{include/william}
\usepackage{csquotes}

\pagestyle{fancy}
\fancyhf{}
\rhead{William E. R.}
\lhead{Documentation | AGV-ACR.AI}
\rfoot{Page \thepage}

\begin{document}

\begin{titlepage}

\topskip0pt
\vspace*{0pt}
\begin{figure}[H]
	\centering
	\includegraphics[width=0.45\textwidth]{images/AGV.png}
\end{figure}
\vspace*{\fill}
\centering 
\Huge Documentation of Work
\vspace{12pt}\\
\small AGV-ACR.AI
\vspace*{\fill}\\
\begin{figure}[H]
	\centering
	\includegraphics[width=0.15\textwidth]{images/logo.png}
\end{figure}
\Large William R. Ebenezaraj
\vspace{2pt}\\
\small \textsc{IIT Kharagpur}\\
\href{mailto:william@iitkgp.ac.in}{william@iitkgp.ac.in}
\vspace*{0pt}
\end{titlepage}

\newpage

\tableofcontents



\newpage

\section{Introduction}
This document is a summary of all work done by \textbf{William R. Ebenezaraj} under AGV-ACR.AI at IIT Kharagpur.

\chapter{Controls Module}
\section{MPC with an Integrated Local Planner}
\subsection{Aim}
To build an MPC which subscribes to a waypoint generator leading to a global nav goal (global planner) and uses MPC strategies to generate an optimal local path and track it accurately.

\subsection{Environment of Implementation}

The following frameworks/packages were employed:
\begin{enumerate}
	\item CARLA 0.9.10.1 running on Ubuntu 20.04
	\item ROS Bridge for 0.9.10.1
	\item ROS Noetic
\end{enumerate}

\section{Model Predictive Control for Lateral Control}

\subsection{Aim}
To build an MPC which subscribes to a waypoint generator leading to a global nav goal (global planner) and uses MPC to track it accurately.

\subsection{Environment of Implementation}

The following frameworks/packages were employed:
\begin{enumerate}
	\item CARLA 0.9.10.1 running on Ubuntu 20.04
	\item ROS Bridge for 0.9.10.1
	\item ROS Noetic
\end{enumerate}

\section{Linear Quadratic Regulator for Lateral Control}

\section{Path Tracking using Pure Pursuit and Stanley}
\subsection{Aim}
To build geometric path trackers that control the lateral actuation of the car (steering) to follow a static path.

\subsection{Environment of Implementation}

The following frameworks/packages were employed:
\begin{enumerate}
	\item Gazebo9 on Ubuntu 18.04
	\item car\_demo package by OSRF
	\item ROS Melodic
\end{enumerate}

\chapter{Embedded Module}
\section{STM32 based Motor Drivers}

\section{Importing SLDPRTs into Gazebo}

\section{Power Distribution Board Design}
\subsection{Aim}
To design a PDB that can handle 

\subsection{Environment of Implementation}

The following frameworks/packages were employed:
\begin{enumerate}
	\item CARLA 0.9.10.1 running on Ubuntu 20.04
	\item ROS Bridge for 0.9.10.1
	\item ROS Noetic
\end{enumerate}

\chapter{Mechanical Module}
\section{Design of URC Wheels}
Wheel designed was chosen to be the one to be implemented on the rover, after extensive refinement for weight reduction and analysis on Static Structural.

\begin{figure}[H]
	\centering
	\includegraphics[width=0.5\textwidth]{images/wheel2.png}
\end{figure}

Other wheels designed for the task:


\section{Design of URC Chassis}
The entire URC was designed by me with inputs from my batchmates at AGV. Sofrware used: Solidworks 2018 and PhotoView 360.

\begin{figure}[H]
	\centering
	\includegraphics[width=0.5\textwidth]{images/Rover.png}
\end{figure}

\section{Design of IGVC Chassis}
\begin{figure}[H]
	\centering
	\includegraphics[width=0.5\textwidth]{images/chass_igvc1.png}
\end{figure}

This task was completed in time, collaborating with Saurabh Mishra. A brief extract from the task report has been included below:

\begin{displayquote}
We propose a chassis with two differentially powered wheels and one free castor wheel. Four motors were noted to offer more traction, but they would consume more power, demanding larger batteries. Omni-wheels were also deliberated upon but were reported to be significantly ineffective for traversal on uneven grass terrain.

The laptop computer controlling the bot will be stationed on the platform, as indicated. The platform is supported by a vertical circular hollow array of colonnades. While beams with a rectangular cross-section are more resilient, the circular cross-section is the best for columns. 

The colonnades will be fixed to the planks using connectors permanently attached to them. This concept avoids the need to use screws in the aft section of the vehicle.
\end{displayquote}

\begin{figure}[H]
	\centering
	\includegraphics[width=0.5\textwidth]{images/chass_igvc2.png}
\end{figure}

\end{document}
